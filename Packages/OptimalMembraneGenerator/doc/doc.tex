\documentclass[12pt]{article}
\usepackage{url}
\usepackage[pdftex,colorlinks=true,urlcolor=blue,filecolor=blue,linkcolor=blue,citecolor=blue,breaklinks=true]{hyperref}
\usepackage{color}

\begin{document}

\newcommand{\loos}{{\sc loos}}
\newcommand{\pyloos}{{\sc PyLoos}}
\newcommand{\namd}{{\sc namd}}
\newcommand{\psfgen}{{\tt psfgen}}
%\newcommand{\omgwtf}{{\bf OMG-WTF}}
\newcommand{\omgwtf}{{\bf OMG}}
\newcommand{\highlight}[1]{\textcolor{red}{#1}}


\begin{center}
\begin{LARGE}
{\bf O}ptimal {\bf M}embrane {\bf G}enerator \\
\end{LARGE}
\begin{Large}
({\bf w}ith  {\bf t}emplate {\bf f}iles) \\
\end{Large}
\vspace*{0.25in}
\begin{large}
The \loos\ membrane builder \\
version 1.0 \\
\end{large}

\vspace*{0.5in}
Dr. Alan Grossfield: alan\_grossfield@urmc.rochester.edu \\
\end{center}

\newpage

\tableofcontents
\newpage

\section{Introduction}
\label{s:intro}

Welcome to the {\bf O}ptimal {\bf M}embrane {\bf G}enerator {\bf w}ith {\bf
t}emplate {\bf f}iles (\omgwtf, for short), the \loos\ membrane building
tool.  The goal of this suite is to facilitate the construction of complex
systems containing lipid bilayers and integral membrane proteins.  The
approach used here merges two previously found in the literature: the
``library of conformations'' approach pushed by Grossfield et al. (PNAS,
2006, 103, 4888-4893) and the INFLATEGRO strategy developed by Tieleman and
others.  The basic algorithm is to start with libraries of previously
generated conformations for the lipids used (generally, one uses neat
simulations to make the libraries, but one can use model building as well
to create an initial library), and pack them around the ``protein'', in
any.  The system's periodic box is initially scaled to 3x its original size in
the plane of the membrane (the normal is always set parallel to the
$z$-axis), allowing the lipids to be randomly placed with relatively few
collisions.  We start by placing the protein (if any) at the origin.
Treating the two leaflets independently, we sequentially add lipids by
picking a conformation from the library, randomly placing it in the
appropriate leaflet and checking for collisions with previously placed
lipids.  This continues until all of the lipids are placed.  Next, the
system is subjected to a series of energy minimizations, each followed by a
small scaling of the positions of the molecules in the plane of the
membrane, gradually shrinking the membrane down to its target size.

\omgwtf\ supports the creation of arbitrarily complex membrane compositions,
including mixtures of lipids and asymmetric bilayers.  The user simply
creates a configuration file that specifies the segments to be included and
which leaflet they should be found in, as well as the number of ions and
water molecules.  

\section{Requirements and Installation}
\label{s:install}

To use the membrane builder, one must first have a working \loos\
installation, with \pyloos\ working.  If you have not downloaded \loos\
yet, you can download it from \url{http://loos.sourceforge.net}, and follow
the instructions included there for building.  

You'll also need to make sure your paths are set up correctly, so that python
can find the correct files.  The \loos\ and \pyloos\ parts of this are best
handled by sourcing the included files {\tt setup.csh} or {\tt setup.sh} (for
{\tt t/csh} and {\tt bash}, respectively).  In addition, you'll need to
append the directory where the python files from this code are installed to
your {\tt PYTHONPATH}.

You will need a set of lipid template files.  These are sets of
lipid conformations, each in the its own coordinate file, which are used in
the construction of new lipid membranes.  We provide a preliminary set for
some of the more common lipids as a separate download, available from
\url{http://sourceforge.net/projects/loos/files/loos\%20material/lipid_library.tgz/download}.   Section \ref{s:library} describes 
how you could go about making your own library of lipid conformations.

Finally, you will need a working \namd\ install, including \namd\ itself and
\psfgen.  At this point, we assume you have the ability to run \namd\
directly on the local machine, rather than via a queuing system.  The jobs
shouldn't take very long (less than 20 minutes for the examples I've done).

\section{Usage}
\label{s:usage}

\omgwtf\ is controlled by a configuration file, given as the first command
line argument; a second optional command line argument specifies the output
directory, where all files will be created (it will use {\tt ./out/} if
nothing is specified).  Everything that is needed, including the location of
the topology and parameter files, the box size, the number of each lipid
type, and the location of the structure libraries are all declared here.

We assume that lipids are laid out in the 1 residue/lipid format 
used in the CHARMM36 parameter files, as opposed to the earlier 3
residue/lipid (chain-headgroup-chain) format often used with CHARMM27.
Furthermore, we adopt the convention that each lipid type in each leaflet
will be its own segment.  So, a pure POPC bilayer would have two segments,
one for the upper leaflet and one for the lower.  While this doesn't affect
the running of the simulation in any way, it can greatly simplify the
analysis by giving an easy way to select specific leaflets without reference
to the coordinates.

The final optimized structure will be {\tt final.pdb}, created in the output
directory.

If you are building a system that contains a protein (or protein-like
molecule), you will need to start out with a working PSF and PDB file for
it; there are too many potential subtleties in creating a protein
(disulfides, unusual charge states, etc) for me to try to programmatically
generate the PSF for you.  The PSF you supply will be included in the
psfgen script {\omgwtf} creates to build the whole system.

While running, {\omgwtf} will write a series of comments to {\tt STDERR};
most of these are simply updates to let you know where it is in the
process, or error messages if it has to stop; it will also issue a warning
if the final system has a net charge, since this is usually not intended.
{\tt STDERR} will also contain output from {\tt psfgen}, which can get
somewhat long.  I'd suggest a quick glance through the output the first few
times you use {\omgwtf}, even if things seem to have worked correctly, just
in case.

In principle, one could take the {\tt final.pdb} generated by {\omgwtf} and
begin production calculations, presumably in the NPT ensemble.  However, we
suggest running a short ($<10$ ns) run in the NPAT (constant area) ensemble
first; depending on how many waters need to be deleted to get to the target
value, the density in the bulk region can be a bit low, and NPAT can be a
bit more stable when adjusting the overall volume.  

\subsection{Format of the configuration file}
\label{ss:config}

The config file is key:value based.  Keys are case-insensitive while values
are case-sensitive.  Keys are assumed to start in the first column of the
line.  Blank lines and lines starting with ``\#'' are ignored as comments.
Unrecognized keys will generate a warning but will not cause an error.

The following keys are used:

\begin{itemize}
    \item {\bf topology filename}: {\bf filename} is the path to the 
                                    \namd-style topology file.
                                    \highlight{REQUIRED}
    \item {\bf parameters filename}: {\bf filename} is the path to the 
                                    \namd-style parameter file. 
                                    \highlight{REQUIRED}
    \item {\bf psf filename}:  {\bf filename} is the path to the \namd-style 
                               psf file that will be generated by \omgwtf.  
                               Typically given as a raw filename without a
                               path. \highlight{REQUIRED}
    \item {\bf segment segname resname num-lip phos-pos phos-name placement library}: 
          \begin{itemize}
             \item {\bf segname} is the name of the segment
             \item {\bf resname} is the residue name of the lipid, e.g. POPC
             \item {\bf num-lip} is the number of lipids in this segment
             \item {\bf phos-pos} is the average distance of the phosphorus
             atom (or whatever atom is used for centering) from the membrane center 
             \item {\bf phos-name} is the atom used to place the lipid; the
                 obvious choice is ``P'', for the phosphorus, but you
                 could use a different interfacial atom to place molecules
                 that don't have phosphorus, e.g. cholesterol
             \item {\bf placement} controls the location of the lipid. ``1''
             indicates the +$z$ leaflet, ``-1'' indicates the lower leaflet,
             and ``0'' indicates a transmembrane species
             \item {\bf library} is the directory containing a library of
             conformations for this lipid
          \end{itemize}
          \highlight{At least 2 segments required (one for each leaflet)}
    \item {\bf water segname resname num-water thickness box-size coords}:
          \begin{itemize}
             \item {\bf segname} is the name of the water segment.  For now,
             we assume that you have fewer than 100,000 waters, so they can
             all be put into 1 segment.  We do not support multiple water
             segments.
             \item {\bf resname} is the residue name of the water, e.g. TIP3
             \item {\bf num-water} is the number of water molecules to
             include
             \item {\bf thickness} the $z$ dimension of the box of water to be
             generated (the $x$ and $y$ dimensions are inherited from the box
             specified for the system as a whole)
             \item {\bf box-size} is the dimension of the water box named by
             the {\bf coords} keyword (presumed to be a cube)
             \item {\bf coords} is a file of coordinates for a
             pre-equilibrated box of water, which will be tesselated to
             create a box of the appropriate size.  This could be a PDB file or 
             a CHARMM-style CRD file ({\tt water\_small.crd} is included in the
             distribution).  Probably any file format supported by
             \loos\ would work here, but only those have been tested.
          \end{itemize}
          \highlight{REQUIRED}
     \item {\bf salt segname resname number}:
          \begin{itemize}
            \item {\bf segname} is the segment to assign this ion to
            \item {\bf resname} is the residue name of the ion, e.g. SOD.  At
            the moment, this is presumed to be a single-atom ion, as opposed
            to something more complex like SO$_4$.
            \item {\bf number} is the number of this ion to include
          \end{itemize}
          \highlight{OPTIONAL}
     \item {\bf protein model-file psf-file water-segment scale-by-molecule}
          \begin{itemize}
            \item {\bf model-file} is a file specifying the contents of the
            ``protein''.  This is presumed to be a PDB file, though other
            structure formats supported by \loos\ should work as well.  The
            program does not translate or reorient this molecule; you
            should have it located as you want it to be in the final
            structure.   The ``protein'' can have as many segments as you
            want, and can contain as many molecules as you want (e.g. a
            protein plus ligand and palmitoylation sites).  If you don't
            have a water segment, use ``NONE'' or some non-existent
            segmentname for the water segment.  

            \item {\bf psf-file} is a \namd-style psf file to create the
            protein.  This will be used to make the other psf files needed
        to run {\namd} during the build process.  \item {\bf water-segment}
            is the name of a segment in the model-file that contains water
            molecules.  The coordinates of these waters will be retained,
            but near the end of construction they will become part of the
            segid specified in the {\bf water} line above.  If you have no
            water segment, put ``NONE''.
            
            \item{\bf scale-by-molecule} determines whether the molecules
                specified by protein get scaled outward in x and y the way
                lipids do.  If you're working with a protein, you probably
                want this set to ``0'', for false.  However, if your
                protein is actually a set of independent molecules (e.g. a
                group of peptide scattered on the surface), you would set
                this to ``1'', indicating true.

          \end{itemize}
          \highlight{OPTIONAL}
     \item {\bf box x y z}: {\bf x, y, and z} are the target dimensions of the
            system periodic box.  \highlight{REQUIRED}
     \item {\bf namd namd-binary}: path to the binary for \namd.
     \highlight{OPTIONAL: defaults to {\tt /opt/bin/namd2}}
     \item {\bf psfgen psfgen-binary}: path to the binary for \psfgen.
     \highlight{OPTIONAL: defaults to {\tt /opt/bin/psfgen}}

\end{itemize}

\subsection{Examples}
\label{ss:examples}

Below are 3 example configuration files, in ascending level of complexity.
Please note that these are demos only, and that I've made no effort to
include correct box sizes or salt concentrations.  They are provided to
give working examples for how to run \omgwtf.  The equivalents of these
files are found in the {\tt example} directory of this distribution; you
will need to update the paths to reflect your particular filesystem, but
otherwise they should work for you out of the box.

To run one of these examples, cd into the appropriate directory (e.g. {\tt
example/popc}), and run the {\omgwtf} giving the config file on the command
line.  For example, the POPC example below would be run as ``{\tt
../../OptimalMembraneGenerator.py popc.cfg test}''; this would run the
{\omgwtf} and put the output in the directory {\tt test}.  On my computer,
the examples take about 10 minutes to run (your mileage may vary).

Note: if you haven't downloaded a lipid library yet, now would be a really
good time to do so.  You can get one from
\url{http://sourceforge.net/projects/loos/files/loos\%20material/lipid_library.tgz/download}.
 


\subsubsection{Pure POPC bilayer}

\begin{quote}
\# Location of the topology and parameter files \\
topology /home/alan/projects/loos-membranes/toppar/top\_build.inp \\
parameters /home/alan/projects/loos-membranes/toppar/par\_build.inp \\

\# Name of the output psf file \\
psf      popc.psf \\

\# Size of the final system periodic box \\
box      75.9  75.9  78.5 \\

\# Segment for the upper leaflet \\
segment TPC       POPC     60    19    P 1      /opt/lipid\_library/popc\_c36
\\
\# Segment for the lower leaflet \\
segment BPC       POPC     60    19    P -1     /opt/lipid\_library/popc\_c36
\\

\# Water and salt \\
\# Note: line-break in the ``water'' line is just in this document
water   BULK     TIP3      8000       50      15.5516 /home/alan/projects/loos-membranes/water\_small.crd \\
salt    SOD       SOD      10 \\
salt    CLA       CLA      10 \\
\end{quote}

\subsubsection{Mixture of POPE and POPG}

\begin{quote}
topology /home/alan/projects/loos-membranes/toppar/top\_build.inp \\
parameters /home/alan/projects/loos-membranes/toppar/par\_build.inp \\
psf      pe\_pg.psf \\
box      75.5  75.5  78.5 \\

\# Upper leaflet \\
segment TPE       POPE     60    19    P 1      /opt/lipid\_library/pope\_c36 \\
segment TPG       POPG     30    19    P 1      /opt/lipid\_library/popg\_c36 \\

\# Lower leaflet \\
segment BPE       POPE     60    19   P -1      /opt/lipid\_library/pope\_c36 \\
segment BPG       POPG     30    19   P -1      /opt/lipid\_library/popg\_c36 \\

\# Salt and water.  Note extra SOD to provide neutrality \\
\# Note: line-break in the ``water'' line is just in this document
water   BULK     TIP3      8000       50      15.5516 /home/alan/projects/loos-membranes/water\_small.crd \\
salt    SOD       SOD      70 \\
salt    CLA       CLA      10 \\

\end{quote}

\subsubsection{Rhodopsin in a mixed bilayer}

\begin{quote}
topology /home/alan/projects/loos-membranes/toppar/top\_build.inp \\
parameters /home/alan/projects/loos-membranes/toppar/par\_build.inp \\
psf      rhod.psf \\
box      75.5  75.5  78.5 \\

\# Starting structure for the protein.  Contains multiple segments, \\
\# including "WAT", which is water \\
protein  all-rhod.pdb all-rhod.psf WAT 0

\# Segname  Resname number 1/-1 (for upper leaflet, lower leaflet) \\
segment TPE   POPE     60    19    P 1      /opt/lipid\_library/pope\_c36 \\
segment TPG   POPG     30    19    P 1      /opt/lipid\_library/popg\_c36 \\
segment BPE   POPE     60    19   P  -1     /opt/lipid\_library/pope\_c36 \\
segment BPG   POPG     30    19   P  -1     /opt/lipid\_library/popg\_c36 \\

\# Note: line-break in the ``water'' line is just in this document
water   BULK     TIP3      8000       50      15.5516 /home/alan/projects/loos-membranes/water\_small.crd \\
salt    SOD       SOD      70 \\
salt    CLA       CLA      10 \\

\end{quote}

\subsection{Algorithm}
\label{ss:algorithm}

The main body of the \omgwtf\ is {\tt OptimalMembraneGenerator.py}.  As
provided, it is written as a single large function, with a series of steps to
build a membrane.  Given how different the additional requirements could be
(multiple bound proteins, peptides in the water region, domains
etc),  I believe that attempting to account for every possibility would
produce a program so complex and cumbersome that it would be too hard to use.
Instead, I'm choosing to produce a relatively simple program that can be
customized to solve each individual problem.

We start with the configuration file (described above in Section
\ref{ss:config}) and the lipid template files (see Section \ref{s:library}).
The program starts by looping over the segments listed in the configuration
file.  For each segment (presumably a lipid species for a particular
leaflet), we pick a random configuration from the library, place it at the
origin, and apply a random rotation about the $z$ axis.  If it's going to be
in the lower leaflet, it is flipped upside down via a 180$^\circ$ rotation
about the $x$-axis.  The molecules is then shifted so phophorus atom is then
placed at the origin, and then molecule is shifted to its new position.  The
$x$ and $y$ coordinates are generated randomly within a box 3 times larger
than the final system target size, while the $z$ position of the phosphorus
is set to be the phosphorus height set in the config file, $\pm 0.5 \AA$
generated randomly; if the lipid is targeted to the lower leaflet, then this
value is negated.  

Before the new lipid placement is accepted, we perform a series of bump
checks, first against the protein (if one was specified in the config file),
then against previously placed lipids; the configuration is rejected its
placement creates at least 3 collisions (defined as a pair of atoms within 3
{\AA} of each other).  Once all lipids have been placed, we write out a PDB
file of the ``scaled'' coordinates, called ``lipid\_only.pdb''.  

Next, we perform a series of scalings and minimizations.  First, the center
of mass of each individual molecule is scaled downward in the $xy$ plane.
The new coordinates are written out as a PDB file called
``lipid\_shrink\_i.pdb'' (where ``i'' is the iteration number), in the
working directory specified on the command line.  This PDB file is then used
to run a 100 step minimization in \namd, with 10 kcal/mol-{\AA}$^2$
restraints applied to each heavy atom only in the $z$ direction; the purpose
of the restraints is to make sure that early collisions don't lead the
membrane to buckle unphysically, instead making lateral displacements more
favorable.  The resulting coordinates are then read back in to the \omgwtf.
After recentering the entire membrane at the origin, the process is repeated
until the system reaches the target size.

Next, we build a box of water using the provided template box, by replicating
the box in each dimension such that its large enough to cover the required
volume (the $x$ and $y$ coordinates match the system ones, while the $z$
dimension is calculated by subtracting 29 {\AA} from the system $z$
dimension.  This is currently hardwired in, and probably should be replaced
by something based on the positioning of the phosphates.  In any event, some
system-specific trial and error is often necessary to get the overall density
right.

At this point, the number of water molecules should be significantly larger
than the target value.  We then superpose the water box onto the previously
placed lipids and protein, and remove waters that clash, where a clash is
defined as a contact between a water oxygen and any heavy atom closer than 1.75
{\AA}.  We verify that we still have enough water, and then proceed to add
the salt ions (if any were requested in the config file) by randomly
selecting and replacing water molecules.

If at this point we still have too many waters, we then proceed to randomly
delete waters to get down to the target value.  Once this is accomplished, we
then merge the waters (if any) specified as part of the protein (if any) into
the water segment (called ``BULK'' in most of our examples).

We construct the final system, with the atoms in the following order: protein
(if any), lipids, salt, then finally water.  We write out the coordinates in
the working directory, in a file called {\tt final.pdb}.  We also write a
psfgen script that will generate the full PSF file for the system, called
``generate\_full\_psf.inp'', and run it for you, creating a PSF with the
output name you specified in the config file.

\section{Lipid Template Files}
\label{s:library}

The lipid template files are libraries of conformations, used to seed the
membranes with a realistic ensemble of structures.  The ``better'' the
ensemble of structures used --- meaning, the closer to that sampled by the
real liquid crystalline membrane --- the less equilibration will be required
before the resulting simulation data is useful.  For this reason, the
templates distributed here are the result of extensive simulations, some
performed with applied tension such that the area/lipid and $^2$H NMR order
parameters match up with experiment.  

\subsection{Templates included in the distribution}
\label{ss:included}

We have included libraries for commonly used lipids.  Or rather, for lipids
commonly used by our group.  In all cases, the lipids use CHARMM36 parameters
and atom naming (although not all of the lipids are explicitly in the
released CHARMM36 files):

\begin{itemize}
    \item POPC
    \item POPE
    \item POPG
    \item SDPC
    \item SDPE
\end{itemize}

\subsection{Making your own templates}
\label{ss:diy}

If you're creating a library for a new lipid, you may or may not have a neat
simulation to use to build the library.  In that case, your best bet is
probably to proceed via 2 steps.  First, ``fake'' a library some other way,
by modifying an existing one.  For example, our original library for POPE was
built from the POPC library, manually replacing the choline methyls with
hydrogens.  Similarly, one could imagine adding or removing carbons from the
acyl chains.  The resulting libraries won't be very good, but one could then
use them to build a few neat bilayers; after running them long enough to
equilibrate them (several hundred nanoseconds may be necessary), the
resulting trajectories can be used to create a new, better library.

Each library file contains a single lipid molecule, with the phosphorus at
the origin and the headgroups pointing in the $+z$ direction.  Other than
these translations, the coordinates should be exactly as extracted from the
simulations, so that the correct overall tilt is retained, and there is no
imposed orientation (e.g. the glycerol backbone isn't aligned along the
$x$-axis).  

\end{document}
